Il lavoro di tesi svolto all’interno del CVD Lab presso l’Università degli Studi Roma Tre ha permesso la realizzazione di un servizio web per la costruzione e fruizione di scene fotorealistiche. Fruizione che viene permessa anche su hardware di basse prestazioni.
\\
In particolare il servizio permette di automatizzare una procedura come il baking delle lightmap, nascondendo completamente all’utente la complessa logica di funzionamento che permette la realizzazione di ombre e luci realistiche, e le ottimizzazioni effettuate sulla scena per renderla fruibile su web. 
\\
Questo consente a qualsiasi tipologia di utilizzatore di effettuare una operazione estremamente complessa come il baking delle lightmap tramite una richiesta al servizio remoto creato, sul quale è installato il software Blender.
\\
Inoltre l’intero servizio è accedibile mediante browser web, non richiedendo alcun attività di installazione o aggiornamento software da parte dell’utente.
\\
Le attività di ricerca e di sperimentazione effettuate hanno quindi permesso di confermare il raggiungimento degli obiettivi di fotorealismo e fluidità prefissati. 
\\
In particolare tali attività di ricerca hanno portato all’ utilizzo del Cycles Render di Blender per permettere l’ottenimento di scene fotorealistiche, ed alla creazione del lightweight render loop, basato su lightmap ed envmap, per permettere una elevata fluidità nella fruizione.
\\
Le attività di sperimentazione effettuate hanno inoltre permesso di confermare la correttezza del mapping dei parametri tra Three.js e Blender, mapping essenziale per permettere a Blender la realizzazione di luci ed ombre.
\\
Il sistema, realizzato secondo le specifiche individuate dovrà essere sottoposto ad una sperimentazione utente per studiarne l’effettiva usabilità. La sperimentazione punta a raccogliere informazioni sull’esperienza d’uso degli utenti, differenti per età e professione, grazie alle quali individuare e risolvere eventuali difficoltà di utilizzo.
\\
Gli obiettivi prefissati per la presente attività di tesi possano pertanto considerarsi raggiunti; ciononostante il sistema realizzato lascia aperto il campo a possibili miglioramenti. 
Inoltre la natura aperta del sistema rende possibile integrarvi nuove funzionalità provenienti anche da contributi esterni, come l’imminente integrazione all’interno del servizio di una widget per la creazione dei muri mediante pianta 2D. 
\\
Per il servizio editor è prevista l’integrazione di un’insieme di funzionalità per la costruzione e l’arredamento di interni, come ad esempio:
\begin{itemize}
\item Un catalogo contenente dei template di arredamento, che l’utente può scegliere in base al tipo di ambiente che intende arredare. L’utente che lo desidera può contribuire al catalogo, inserendo un template di arredamento da lui creato.
\item Creazione di sessioni di lavoro condivise tra più utenti, i quali possono quindi partecipare alla costruzione di un’unica scena.
\end{itemize}
A queste si aggiungono tutto un’insieme di attività volte migliorare l’usabilità del servizio, come la possibilità di alternare visione prospettica e ortografica all’interno della scena, o l’aggiunta di ulteriori shortcut da tastiera.
\\
Per il servizio di navigazione è prevista l’integrazione di nuove funzionalità, come ad esempio:
\begin{itemize}
\item Estendere la compatibilità del servizio con le nuove tecnologie di realtà aumentata e virtuale. Tali tecnologie saranno prima valutate, e se ritenute adatte al contesto di utilizzo, integrate. 
\item Calcolo di un percorso di navigazione prefissato all’interno della scena, con la possibilità di realizzare dei tour virtuali dell’ambiente.
\end{itemize}

Per il servizio di baking è prevista l’installazione dello stesso su di un cluster di calcolatori, in modo da poter gestire simultaneamente un numero maggiore di processi di bake. 
Altra attività prossima all’implementazione sarà l’aggiunta delle utenze all’interno del servizio. In particolare ogni utente potrà registrarsi e disporre di un’area personale dove poter consultare lo storico dei bake da lui richiesti, o le richieste in attesa di essere elaborate; per quest’ultime in particolare il servizio già ne permette la visualizzazione, ma in modalità condivisa.
\\ 
Inoltre sono previste delle attività di ottimizzazione volte a ridurre il quantitativo di memoria richiesto dal baking delle lightmap. In particolare si potrebbe pensare di decomporre la scena in sezioni più piccole, dove ogni oggetto influenza o è influenzato dal baking solo all’interno della sezione. Ognuna di queste può quindi essere renderizzata separatamente, riducendo quindi il consumo di memoria da parte del processo di rendering.
\\
In conclusione può considerarsi ultimata la realizzazione un sistema che nella sua interezza presenta caratteristiche innovative.
\\
L’innovazione risiede proprio nella possibilità di creare, rendere fotorealistica e navigare una scena 3D mediante servizio web, accessibile da qualunque dispositivo munito di browser e connessione ad internet. Grazie al sistema realizzato queste funzionalità, normalmente fruibili su calcolatori ad alte prestazioni, sono permesse anche su dispositivi di fascia medio/bassa.
\section{Caratteristiche di design dell’Editor}
\label{sec:chapter_creazione_scena_design}
Nel corso di questo capitolo sono state descritte le principali funzionalità offerte dall’editor, menzionando nell’introduzione come queste siano organizzate utilizzando le tecnologie Web Components.
\\
Seguendo questa caratteristica di design, dettata dalla filosofia di Polymer, all’interno dell’editor tutto è un elemento.
\\
In particolare l’applicazione è organizzata in componenti, i quali a loro volta sono organizzati in altri componenti.
\\
I componenti principali all'interno dell'editor sono il \emph{pannello delle opzioni} (o \emph{sidebar} ), e l’\emph{editor threejs} . Inoltre vi è anche un componente secondario, per il catalogo di prodotti.
\\
Il pannello delle opzioni è diviso in altri componenti, ossia pannelli, ognuno per una funzionalità specifica; questi sono il \emph{pannello main} , il \emph{pannello scena} , il \emph{pannello per gli inserimenti} , e il \emph{pannello contestuale all’oggetto selezionato} .
\\
\\
Il pannello main si decompone nel \emph{pannello dei file} , e in quello \emph{di progetto} . 
Il pannello dei file racchiude tutte le funzionalità per l’importazione e il salvataggio di oggetti o scene, e per l’apertura del catalogo di prodotti.
\\
Il pannello di progetto si decompone a sua volta nel \emph{pannello di bake} , con le relative funzionalità di bake già discusse nel paragrafo \ref{sec:chapter_creazione_scena_funzionalita_editor_bake} , e i pannelli per la gestione della \emph{griglia} e della \emph{modalità di trasformazione} degli oggetti, e un’ultimo pannello con le \emph{statistiche} di scena (come il numero di vertici contenuti).
\\
\\
Il pannello scena contiene la lista degli oggetti all’interno della scena. 
\\
\\
Il pannello per gli inserimenti si divide in quattro diversi pannelli, che differiscono nel tipo di oggetti che permettono di inserire: geometrie, luci, camere, skybox. Ogni pannello racchiuderà le funzionalità per inserire oggetti di uno dei quattro tipi sopra citati.
\\
\\
Per quanto riguarda l’ultimo pannello, ovvero quello contestuale, è organizzato in maniera differente a seconda del modello ThreeJS selezionato.
\\ 
Se il modello è una mesh, allora il pannello si decompone in due ulteriori pannelli: \emph{Mesh} e \emph{Material} . Il pannello Mesh contiene tutte le funzionalità per trasformare la geometria della mesh, o configurare alcune impostazioni di bake. Il pannello Mesh contiene invece le funzionalità per modificare i materiali o texturizzare il modello.
\\
Se invece il modello selezionato è una luce, allora il pannello sarà decomposto nei pannelli \emph{Mesh} e \emph{Light} , il primo contenente le funzionalità per muovere la fonte luminosa all’interno della scena, il secondo per modificare le proprietà di illuminazione, come intensità luminosa o colore. 
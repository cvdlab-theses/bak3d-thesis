\section{Web component e Polymer}
\label{sec:chapter_tecnologie_abilitanti_webcomp_poly}

Al giorno d’oggi le applicazioni web sono molto complesse: è fondamentale trovare il modo corretto di suddividere lo sviluppo dell’applicazione in più componenti non sovrapposte, in modo che il team di sviluppo operi in maniera più efficiente. 
Ogni componente del sistema deve garantire la proprietà di isolamento, per nascondere la propria complessità agli altri componenti.
Un buon isolamento tra componenti rende il software più manutenibile e semplice da sviluppare.
Inizialmente la complessità delle applicazioni web veniva gestita lato server, isolando l’applicazione in più pagine web; questo obbligava l’utente a dover navigare molte pagine sul browser.
Con l’avvento di tecniche come AJAX, che hanno semplificato la realizzazione di single page applications, gli utenti hanno cominciato a fruire di una esperienza d’uso più fluida, simile a quella che si ha con le applicazioni desktop. 
La logica delle applicazione lato client, che spesso è anche più complessa di quella lato server, può essere isolata all’interno di una singola pagina web o documento mediante i web component.

I web component sono un’ insieme di tecnologie prodotte da  Google e approvate dal W3C che consentono la creazione di widget, ovvero elementi html che includono logica di funzionamento e stile, che possono essere importate in documenti o applicazioni web. Da una widget viene separata ogni responsabilità che ha a che fare con altre widget; di conseguenza ogni elemento non dipende dai dettagli implementativi degli altri elementi. Sono possibili dipendenze solo tra le interfacce esposte dalle widget. 
Assegnare le responsabilità in questo modo permette al modello a componenti di soddisfare la proprietà di incapsulamento.
Inoltre la fruibilità di elementi html facilmente integrabili nell’applicazione permette alle componenti di interagire in modo semplice tra loro; in questo modo è soddisfatta anche la proprietà di interoperabilità tra componenti.
Questi web component sono parte del browser, quindi non hanno bisogno di librerie esterne per essere usati, ma basta includerli aggiungendo un tag nella pagina html. 
Una widget è portabile e gode di un elevata riusabilità; inoltre all’interno di una widget è possibile effettuare tutto ciò che è consentito dai linguaggi html, css e JavaScript. 
L’obiettivo principale dei web component è cambiare il modo in cui le applicazioni web vengono create, consentendo agli sviluppatori di estendere il vocabolario HTML creando nuovi elementi HTML riusabili. Questa aggiunta permette agli sviluppatori di interfacciarsi a più alto livello con lo sviluppo di applicazioni web, definibili mediante aggregazione di più componenti.
Quando si parla di web components in verità si parla di quattro tecnologie differenti, le quali insieme forniscono tutti gli strumenti necessari per creare i propri componenti.
Tali tecnologie sono: Custom Elements, Templates, HTML Imports e Shadow DOM. 

\subsection{Custom elements}
\label{sec:chapter_tecnologie_abilitanti_custom_elements}

La tecnologia custom element permette di creare nuovi elementi HTML all’interno del DOM del browser; ogni elemento racchiude logica di funzionamento, stile CSS e un tag HTML necessario per importare l’elemento nelle applicazioni web. 
Nonostante questi elementi facciano parte dei web components, possono essere utilizzati anche come entità separate. Uno dei motivi per cui è stata creata la tecnologia Custom Elements è la possibilità di introdurre le lifecycle callbacks, che consentono di associare specifici comportamenti ad un elemento HTML durante il suo ciclo di vita. Ad esempio è possibile istruire un elemento HTML sul comportamento che deve assumere nel momento in cui viene inserito nel DOM, e su quello che deve assumere quando viene rimosso dal DOM. 
Per registrare un nuovo elemento HTML nel browser si usa il metodo \texttt{Document.registerElement()}, e il nuovo elemento utilizzerà di default l’interfaccia HTMLElement; se invece l’elemento non viene registrato sul browser, userà l’interfaccia HTMLUnknownElement. E’ anche possibile creare un elemento HTML come estensione di un elemento nativo, come \texttt{<button>}; in questo caso però non sarà possibile includere l’estensione nel DOM come elemento a se stante, ma sempre insieme all’elemento nativo che estende; ad esempio l’inclusione sarà \texttt{<button is=”my button”>} e non \texttt{<my-button>} .
I vantaggi apportati dalla tecnologia custom elements sono:
\begin{enumerate}
\item Definizione e creazione di nuovi elementi HTML/DOM.
\item Estendere la logica funzionale di elementi nativi mediante nuovi elementi.
\item Racchiudere tutte le funzionalità di un nuovo elemento in un tag.
\item Estendere le API degli elementi del DOM nativi.
\item Estrema utilità nelle Single Page Applications.
\end{enumerate}
A questo punto verrà mostrato come è possibile creare un nuovo custom element utilizzando ad esempio le classi di JavaScript 6.
Per prima cosa viene definita la classe, la cui struttura è tipicamente la seguente:
\begin{lstlisting}
class MyButton extends HTMLElement {    
/* E possibile estendere 
   qualsiasi altro elemento o classe */
/* Costruisco le proprieta dell'elemento 
   esteso, ovvero HTMLElement */     
   constructor() {          
      super();                         
   }   								  
/* Qui vengono definite quelle necessarie 
   tra le quattro lifecycle callbacks offerte:
   createdCallback(){ . . } 
   attachedCallback(){ . . } 
   detachedCallback(){ . . } 
   attributeChangedCallback() { . . } */
/* Qui vengono definiti i metodi getters e setters:
   set ...{ . . } 
   get ...{ . . } */
}
\end{lstlisting}
Come discusso in precedenza, le lifecycle callback risultano uno strumento utile nella definizione di un custom element, Lo sviluppatore può decidere di implementare una delle quatto callback disponibili:
\begin{itemize}
\item \emph {createdCallback} - Il comportamento dell’elemento quando viene registrato.
\item \emph {attachedCallback} - Il comportamento dell’elemento quando viene inserito nel DOM.
\item \emph {detachedCallback} - Il comportamento dell’elemento quando viene rimosso dal DOM.
\item \emph {attributeChangedCallback(attributeName)} - il comportamento dell’elemento quando un suo attributo viene modificato.
\end{itemize}
\emph{createdCallback} viene invocata in modo sincrono alla creazione dell’elemento, le altre callback sono invece chiamate in modo asincrono. Le callback asincrone solitamente sfruttano \texttt{MutationObserver} per mettersi in ascolto dei cambiamenti sul DOM; questo consente alle callbacks di essere invocate prima di altri elementi come layout, stile, o altri eventi; in questo modo si evitano problemi, come il caricamento di contenuto senza stile, prima che la callbacks abbiano modo di reagire alle modifiche sul DOM.



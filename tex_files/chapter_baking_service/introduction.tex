In questo capitolo verrà mostrato nel dettaglio il servizio dedicato al processamento del baking delle lightmap su richiesta dell’utente. 
\\
Nel paragrafo \ref{sec:chapter_architettura_sistema_il_servizio_baking} si è descritto in generale il servizio, individuandone i punti focali nella gestione diversificata delle richieste utente in base al loro tipo, e nella pipeline di elaborazione che viene eseguita a fronte di una richiesta di baking delle lightmap. In questo capitolo verranno dettagliati tali punti, discutendo le soluzioni scelte a fronte di ogni problema. Inoltre saranno presenti riferimenti alle tecnologie utilizzate per implementare determinate soluzioni. 
\\
Attualmente il servizio di bake viene erogato da un calcolatore collegato alla rete.
Sul calcolatore è implementato un web server; questo web server è configurato per eseguire specifiche operazioni in base al tipo di richieste che cattura. Per riconoscere il tipo della richiesta, il server guarda all’URL e al metodo della richiesta. Il tipo di richiesta principale che il server cattura è una richiesta di bake; l’operazione corrispondente consiste in una serie di pre e post elaborazioni, che fanno da contorno all’operazione più importante, ovvero il processo di baking. Per invocare tale processo, è necessario che sul calcolatore sia installato un software di grafica 3D abilitato alla generazione di ligthmap, che in questo caso è Blender. La modalità con cui avviene l’istanziazione di Blender, per richiedere un bake, verrà spiegato nel prossimo paragrafo. Ovviamente trattandosi della procedura computazionalmente più onerosa, si è scelto di installare il servizio su di un calcolatore munito di hardware di accelerazione grafica performante. 
\\
Oltre alle richieste di bake di una scena, il web server è in grado di elaborare richieste per ottenere la lista dei bake in attesa o in esecuzione sul servizio, richieste per conoscere lo status di completamento di un bake, o richieste per annullare un bake.
\\ 
Nel paragrafo \ref{sec:chapter_architettura_sistema_il_servizio_baking} si è accennato al fatto che il server tenga memoria delle sue azioni così che, in caso fallimento da parte del client, esso sia in grado di ricevere il risultato e di effettuare tutte le operazioni sopra citate.
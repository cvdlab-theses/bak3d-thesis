La grafica 3D negli ultimi anni ha raggiunto traguardi tali da permettere la creazioni di scene virtuali ad elevato livello di realismo.
\\
L’evoluzione delle tecniche di grafica 3D hanno permesso di riprodurre fedelmente la realtà, ricreando persone ed ambienti estremamente credibili. Tali tecniche vengono utilizzate soprattutto nell’architettura e nell’industria cinematografica e videoludica.
\\
I videogiochi, in particolare, permettono all’utente di immedesimarsi in prima persona all’interno di credibili ambienti virtuali. Inoltre l’utilizzo dei sensori virtuali nei giochi ha permesso di annullare quasi completamente la percezione che l’utente ha dell’ambiente reale, dandogli la sensazione di trovarsi all’interno di un luogo differente.
\\
\\
Il particolare interesse legato al mondo videoludico ed alle tecnologie di realtà virtuale appena citate, unito alla volontà di approfondire i software e le tecniche di grafica 3D che permettono la realizzazione di ambienti fotorealistici, ha portato alla scelta di un lavoro di tesi che affrontasse tali argomenti.
\\
Il lavoro è stato svolto in collaborazione con Edoardo Carra presso il CVD Lab nell’Università degli Studi di Roma Tre. Lavoro che ha inoltre permesso di approfondire un ulteriore argomento di interesse come la programmazione su web.
\\
L’obiettivo è consistito nella realizzazione di un servizio web che permettesse la costruzione e la fruizione di scene fotorealistiche anche su hardware di basse prestazioni.
Un servizio che inoltre fosse accessibile da qualsiasi dispositivo mediante browser web e che non richiedesse alcuna attività di installazione o aggiornamento software da parte dell’utilizzatore.
\\
La natura web del servizio comporta però gravi problemi di fluidità quando si prova a rendere fotorealistica una scena, fluidità che soprattutto su hardware di fascia bassa rende inutilizzabile il servizio.
\\
L’attività di ricerca svolta durante il lavoro di tesi ha permesso di individuare le caratteristiche che maggiormente incidono sulla fluidità consentendo la creazione di una procedura in grado di rendere le scene create dal sistema fotorealistiche e fruibili su web.
Procedura valida sotto una determinata precondizione abilitante che verrà dettagliata nell’elaborato.
\\
Inoltre il servizio permette di automatizzare questa procedura, nascondendone all’utente la complessa logica di funzionamento su cui si basa.
\\
Il servizio nel dettaglio offre gli strumenti che consentono all’utilizzatore di creare, rendere fotorealistica e navigare una scena. 
La creazione viene effettuata tramite un servizio editor, la resa fotorealistica tramite il servizio di bake e la fruizione della scena tramite il servizio navigatore.
\\
L’editor offre tutti gli strumenti che permettono la realizzazione su web di scene 3D con la possibilità, tra le altre, di caricare oggetti, posizionarli nell’ambiente ed inserire luci. 
In particolare, in questo lavoro di tesi, ha permesso la costruzioni di abitazioni 3D arredate.
\\
Il servizio di bake rappresenta un servizio remoto a cui viene affidata la scena creata nell’editor e si occupa di effettuare la procedura per renderla fotorealistica ed ottimizzata per essere utilizzata in ambiente web. In particolare il servizio permette la creazione di luci ed ombre estremamente realistiche.
\\
La scena costruita in questo modo viene affidata al servizio navigatore che permette la fruizione di essa in ambiente web.
\\
Il navigatore  permette di sopperire i limiti dovuti dall’utilizzo di immagini statiche della scena virtuale che consentono di valutare l’ambiente creato solamente da punti di vista fissi.
In particolare permette all’osservatore (controllato dall’utente) di passeggiare virtualmente all’interno dello spazio creato, dando la sensazione di visitare realmente la scena realizzata, in maniera molto simile a come viene effettuato nei videogiochi.
\\
Questo consente all’utente di valutare in maniera estremamente accurata l’ambiente virtuale, permettendo di osservare a piacimento ogni singolo dettaglio presente in essa.
\\
Un sistema di questo tipo, nella sua interezza, presenta caratteristiche innovative in quanto permette di creare, rendere fotorealistica e navigare una scena 3D mediante servizio web accessibile da browser. Inoltre queste funzionalità, normalmente fruibili su hardware ad alte prestazioni, vengono permesse anche su dispositivi poco performanti.
\\
Un servizio del genere può essere utilizzato per permettere la comunicazione fra il committente e l’esecutore in maniera più semplice, in quanto agevolata dalla comprensibilità del prodotto in 3D. Permettendo di mostrare in relativo breve tempo progetti in tre dimensioni che verranno poi realizzati realmente anche dopo elevate distanze temporali.
\\
Un architetto potrebbe ad esempio utilizzare il sistema per la creazione della struttura di un appartamento 3D in cui sono presenti le informazioni estremamente realistiche di luci ed ombre.
\\
Una volta creata la scena virtuale l’utente, tramite il servizio di navigazione, potrebbe visitare l’appartamento ancora prima che esso sia stato realizzato.
\\ 
Gli acquirenti in questo modo otterrebbero una idea precisa dello spazio disponibile e di quanto le stanze siano illuminate dalla luce solare. Informazione fondamentale anche per l’architetto durante la progettazione dell’abitazione.
\\
Questo sistema permetterebbe all’utente di navigare la scena accedendo solamente al servizio in rete, non costringendolo ad installare alcun software di grafica 3D.
Inoltre l’utilizzo di visori di realtà virtuale, che permettono la visione stereoscopica 3D dell’ambiente, consentirebbero all’utente di ottenere una migliore percezione dello spazio, grazie alle informazioni di profondità.
\\
Il sistema potrebbe anche essere utilizzato da un arredatore di interni al fine di mostrare al cliente la possibile disposizione di mobili e di decorazioni all’interno di un ambiente virtuale che rispecchia la sua abitazione. 
\\
Anche in questo caso il servizio permetterebbe all’utente di farsi una idea estremamente precisa di come sarà il risultato finale.
\\
Inoltre potrebbe essere utilizzato da organizzatori di musei o mostre per ricreare virtualmente  l’ambiente da visitare o una porzione di esso, rendendola accessibile da ogni utente sul web. 
\\
Gli utenti anche in questo caso navigherebbero la scena, ottenendo una idea chiara di quello che offre ancor prima di averla visitata. Potrebbe inoltre essere utilizzata per mostrare agli utenti porzioni di museo momentaneamente chiuse.
\\
Anche in questo caso l’utilizzo dei visori di realtà aumentata permetterebbero di dare all’utilizzatore l’illusione di trovarsi realmente all’interno del museo.
\\
\\
Ogni aspetto del lavoro di tesi è stato interamente svolto in collaborazione con il collega Edoardo Carra ed i due elaborati differiscono per la modalità con cui viene trattato un argomento.
In particolare nel presente elaborato verrà descritto accuratamente il servizio di navigazione e trattato con minore dettaglio il servizio di editor.
\\
Nell' elaborato viene inizialmente effettuata una panoramica generale delle tecniche e tecnologie abilitanti per la comprensione degli argomenti affrontati.
\\
Successivamente viene descritta la modalità con cui le tecniche scelte sono state utilizzate al fine di rendere la scena fotorealistica ed in quale condizioni possono essere applicate.
\\
Viene poi presentata l’architettura del sistema e vengono discussi: l’editor, il servizio di bake ed il navigatore. In particolare nell’elaborato verranno esaminati accuratamente il servizio di bake e di navigazione.
\\
Infine vengono presentati e discussi i risultati ottenuti ed i possibili sviluppi futuri del sistema realizzato.

